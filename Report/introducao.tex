\chapter{Introdução}
A previsão meteorológica é importantíssima para a tomada de decisões no âmbito da agricultura. Com ela é possível melhorar a caracterização dos acidentes meteorológicos com maior risco para as culturas em causa. 
Com a cada vez maior inconstantibilidade do tempo, torna-se fulcral ter uma melhor capacidade de prever as condições meteorológicas para uma melhorar a preparação contra acidentes meteorológicos.
\\Para se conseguir uma previsão mais correcta, é muito importante ter uma quantidade considerável de dados observadoos para diminuir a margem de erro das mesas. Uma das formas para se obter mais dados observados, será a utilização de estações meteorológicas em mais locais do território português.
\\Com esta oportunidade da parceria com a FAABA e o IM torna-se mais simplificada esta tarefa.
 

\section{Objectivos}
Este projecto pretende criar uma solução que sirva de ponte entre os agricultores do Baixo Alentejo e o Instituto de Meteorologia para que se possa apresentar as previsões meteoreológicas, avisos e alertas aos agricultores para lhes facilitar o processo de decisão no que concerne à sua actividade.

\section{Estrutura do Relatório}
Neste trabalho é apresentada uma solução para implementação de estações metereológicas, a recolha dos respectivos dados, armazenamento e envio ao Instituto de Meteorologia.
Na segunda parte do trabalho é apresentada a forma como obtemos as previsões meteorológicas e climatologia calculados pelo IM.
Por fim é apresentada a interface criada para a aplicação que apresentará aos utilizadores não só os dados observados por cada uma das estações, mas também as previsões calculadas pelo IM.